
\pgfmathsetmacro{\cardwidth}{6}
\pgfmathsetmacro{\cardheight}{9}

\pgfmathsetmacro{\stripwidth}{1.2}
\pgfmathsetmacro{\strippadding}{0.1}
\pgfmathsetmacro{\textpadding}{0.15}
\pgfmathsetmacro{\ruleheight}{0.3}

\newcommand{\carddebug}{
    \draw [step=0.2,help lines] (0,0) grid (\cardwidth,\cardheight);
}

\def\shapeCard{(0,0) rectangle (\cardwidth,\cardheight)}
\tikzset{
    %   runde Ecken für die Karten
    cardcorners/.style={
        rounded corners=0.2cm
    },
    %   runde Ecken für die "Fähnchen"
    elementcorners/.style={
        rounded corners=0.1cm
    },
    %   Schlagschatten für die "Fähnchen"
    stripshadow/.style={
        drop shadow={
            opacity=.5,
            shadow,
            color=black
        }
    },
    %   Style für die "Fähnchen"
    strip/.style={
        elementcorners,
        stripshadow
    },
    %   Bild für das Kartenmotiv
    cardimage/.style={
        path picture={
            \node[below=-1.5mm] at (0.5*\cardwidth,\cardheight) {
                \includegraphics[width=\imagewidth cm]{#1}
            };
        }
    },
}

\newcommand{\cardborder}{
    \draw[lightgray,cardcorners] \shapeCard;
}

\newcommand{\cardreverse}{
  \begin{tikzpicture}
  \node [scale=2.5] at ({0.5*\cardwidth}, 0.5*\cardheight) {
  %LaTeX with PSTricks extensions
%%Creator: inkscape 0.91
%%Please note this file requires PSTricks extensions
\psset{xunit=.5pt,yunit=.5pt,runit=.5pt}
\begin{pspicture}(100,125)
{
\newrgbcolor{curcolor}{0 0 0}
\pscustom[linestyle=none,fillstyle=solid,fillcolor=curcolor]
{
\newpath
\moveto(46.154,121.903)
\curveto(48.53,122.023)(50.913,122.028)(53.291,121.935)
\curveto(58.662,121.57)(63.981,120.378)(68.96,118.32)
\curveto(80.117,113.785)(89.502,105.063)(94.831,94.262)
\curveto(98.163,87.544)(100.003,80.061)(99.99,72.555)
\curveto(100.204,62.115)(96.864,51.671)(90.813,43.175)
\curveto(87.233,38.176)(82.769,33.777)(77.601,30.432)
\lineto(77.481,30.474)
\curveto(75.809,32.579)(74.537,35.018)(73.868,37.628)
\curveto(73.374,39.196)(73.117,40.825)(72.806,42.436)
\curveto(72.425,43.422)(71.218,43.398)(70.506,44.015)
\curveto(69.288,44.956)(68.404,46.462)(68.376,48.015)
\curveto(68.569,48.111)(68.82,48.093)(68.948,47.898)
\curveto(69.391,47.377)(69.689,46.731)(70.224,46.285)
\curveto(71.717,45.005)(73.74,44.435)(75.685,44.554)
\curveto(77.098,44.742)(78.583,44.552)(79.934,45.103)
\curveto(82.445,45.977)(84.336,47.921)(86.406,49.5)
\curveto(87.16,50.009)(87.828,50.632)(88.517,51.223)
\curveto(88.544,51.279)(88.597,51.387)(88.622,51.442)
\curveto(88.15,51.525)(87.694,51.321)(87.232,51.25)
\curveto(85.961,51.031)(84.688,50.803)(83.4,50.7)
\curveto(81.509,50.388)(79.585,50.314)(77.671,50.392)
\curveto(76.992,50.464)(76.295,50.49)(75.651,50.738)
\curveto(74.352,51.175)(73.363,52.186)(72.454,53.169)
\curveto(70.484,55.428)(69.042,58.072)(67.308,60.504)
\curveto(66.455,61.657)(65.465,62.776)(64.156,63.416)
\curveto(63.295,63.789)(62.375,64.024)(61.442,64.117)
\curveto(61.023,64.104)(60.813,64.529)(60.522,64.76)
\curveto(59.813,65.473)(58.782,65.621)(57.882,65.967)
\curveto(57.337,66.187)(56.746,66.22)(56.167,66.233)
\curveto(53.505,66.63)(50.756,66.245)(48.131,66.937)
\curveto(47.038,67.286)(46.095,67.963)(45.145,68.585)
\curveto(43.823,69.444)(42.594,70.444)(41.214,71.213)
\curveto(39.185,72.101)(37.97,74.082)(36.151,75.277)
\curveto(34.942,76.391)(33.506,77.215)(31.997,77.855)
\curveto(31.021,78.372)(29.92,78.988)(29.523,80.084)
\lineto(29.565,80.231)
\curveto(30.349,80.157)(31.295,79.864)(31.962,80.427)
\curveto(31.98,80.67)(31.795,80.85)(31.668,81.032)
\curveto(31.213,81.586)(30.785,82.171)(30.221,82.623)
\curveto(29.442,83.266)(28.373,83.482)(27.684,84.249)
\curveto(26.647,85.465)(26.267,87.118)(25.144,88.272)
\curveto(24.549,88.888)(23.581,88.73)(22.924,89.24)
\curveto(21.57,90.16)(20.546,91.46)(19.389,92.599)
\curveto(18.787,93.162)(18.535,93.97)(18.05,94.619)
\curveto(17.868,94.859)(17.525,94.736)(17.341,94.568)
\curveto(16.887,94.17)(16.506,93.69)(16.162,93.193)
\curveto(15.828,92.682)(15.96,92.048)(15.952,91.476)
\curveto(15.413,91.583)(14.659,91.783)(14.362,91.153)
\curveto(14.014,90.326)(14.08,89.3)(14.487,88.496)
\curveto(15.526,86.55)(15.609,84.29)(16.372,82.246)
\curveto(16.659,81.231)(17.228,80.224)(17.06,79.14)
\curveto(16.533,78.199)(16.18,77.158)(16.123,76.074)
\curveto(15.989,73.829)(16.173,71.574)(16.512,69.354)
\curveto(16.934,67.497)(17.637,65.713)(18.469,64.001)
\curveto(19.179,62.322)(20.441,60.978)(21.569,59.576)
\curveto(22.481,58.412)(23.704,57.545)(24.668,56.428)
\curveto(25.263,55.753)(26.057,55.225)(26.48,54.413)
\curveto(27.421,51.845)(28.455,49.286)(29.943,46.978)
\curveto(31.722,43.961)(32.081,40.324)(31.769,36.892)
\curveto(31.699,36.553)(31.668,35.964)(31.191,35.99)
\curveto(30.175,36.019)(29.16,36.145)(28.15,36.265)
\curveto(27.814,36.277)(27.428,36.348)(27.155,36.097)
\curveto(26.228,35.425)(25.303,34.749)(24.373,34.082)
\curveto(24.079,33.853)(23.691,33.973)(23.354,33.973)
\curveto(21.739,34.047)(20.118,34.022)(18.519,34.284)
\curveto(17.924,34.399)(17.281,34.416)(16.746,34.731)
\curveto(11.831,39.127)(7.752,44.468)(4.896,50.417)
\curveto(1.753,57.004)(-0.031,64.281)(0.01,71.593)
\curveto(-0.222,82.414)(3.374,93.237)(9.854,101.893)
\curveto(15.7,109.741)(23.884,115.843)(33.111,119.117)
\curveto(37.307,120.633)(41.716,121.526)(46.154,121.903)
\closepath
}
}
{
\newrgbcolor{curcolor}{0 0 0}
\pscustom[linestyle=none,fillstyle=solid,fillcolor=curcolor]
{
\newpath
\moveto(42.94,49.27)
\curveto(43.894,50.397)(45.449,50.848)(46.888,50.752)
\curveto(47.515,50.803)(48.149,50.553)(48.57,50.084)
\curveto(48.795,49.795)(48.893,49.433)(49.031,49.097)
\curveto(49.826,47.049)(50.544,44.957)(51.585,43.015)
\curveto(52.281,41.928)(53.228,41.03)(54.183,40.174)
\curveto(55.366,39.159)(56.564,38.157)(57.849,37.271)
\curveto(58.4,36.875)(59.063,36.574)(59.399,35.946)
\curveto(59.954,34.961)(59.986,33.794)(60.181,32.703)
\curveto(60.231,32.17)(60.437,31.639)(60.321,31.099)
\curveto(59.959,31.055)(59.746,31.515)(59.37,31.48)
\curveto(57.764,31.398)(56.192,31.007)(54.59,30.867)
\curveto(54.083,30.846)(53.525,30.594)(53.056,30.887)
\curveto(51.742,31.581)(50.478,32.366)(49.172,33.076)
\curveto(48.72,33.347)(48.17,33.23)(47.673,33.213)
\curveto(46.731,33.079)(45.772,33.009)(44.823,33.107)
\curveto(44.332,33.243)(43.839,33.376)(43.346,33.503)
\curveto(42.735,33.673)(42.848,34.458)(42.81,34.949)
\curveto(42.732,35.6)(43.029,36.188)(43.26,36.774)
\curveto(43.503,38.559)(43.211,40.35)(43.114,42.134)
\curveto(42.898,43.887)(42.845,45.654)(42.799,47.417)
\curveto(42.75,48.028)(42.6,48.716)(42.94,49.27)
\closepath
}
}
{
\newrgbcolor{curcolor}{0 0 0}
\pscustom[linestyle=none,fillstyle=solid,fillcolor=curcolor]
{
\newpath
\moveto(35.896,50.555)
\curveto(36.49,50.808)(37.442,50.803)(37.63,50.036)
\curveto(38.518,46.623)(39.002,43.12)(39.55,39.64)
\curveto(39.761,38.069)(39.595,36.448)(38.993,34.975)
\lineto(38.869,34.887)
\curveto(37.662,35.089)(36.48,35.413)(35.286,35.673)
\curveto(34.857,35.716)(34.894,36.225)(34.841,36.54)
\curveto(34.816,37.464)(34.774,38.394)(34.944,39.307)
\curveto(35.016,40.586)(34.912,41.876)(35.101,43.148)
\curveto(35.106,44.624)(35.403,46.085)(35.295,47.562)
\curveto(35.242,48.208)(34.586,48.669)(34.661,49.341)
\curveto(34.808,49.934)(35.345,50.35)(35.896,50.555)
\closepath
}
}
{
\newrgbcolor{curcolor}{0 0 0}
\pscustom[linestyle=none,fillstyle=solid,fillcolor=curcolor]
{
\newpath
\moveto(61.812,42.426)
\curveto(62.026,42.527)(62.255,42.467)(62.458,42.369)
\curveto(63.971,41.727)(65.498,41.098)(66.908,40.247)
\curveto(67.429,39.936)(67.852,39.485)(68.211,38.999)
\curveto(69.501,37.305)(70.659,35.507)(71.655,33.622)
\curveto(72.6,31.829)(73.059,29.779)(72.93,27.754)
\curveto(70.142,28.123)(67.416,28.844)(64.655,29.363)
\curveto(64.399,29.437)(63.984,29.423)(63.951,29.771)
\curveto(63.655,31.863)(63.773,33.993)(64.177,36.06)
\curveto(64.33,36.789)(63.636,37.224)(63.185,37.653)
\curveto(61.955,38.845)(61.756,40.673)(61.747,42.296)
\lineto(61.812,42.426)
\closepath
}
}
\end{pspicture}
};
  % \cardborder
  \node[anchor=north west, align=left,\backgroundfontcolor,font=\small] at ({0.4}, 1.4){
    \\{\normalfont\mdseries \footnotesize Wolf by Lee Mette from the Noun Project}};
  \draw[white,cardcorners] \shapeCard;
  \end{tikzpicture}
}

\newcommand{\materialcard}[5]{
\begin{tikzpicture}
\begin{scope}[even odd rule]
% Draw Card shape and background
  \cardborder
  \clip[cardcorners] \shapeCard;
  \fill[\blackbackgroundcolor] (0,0) rectangle (\cardwidth,\cardheight);%
  % \carddebug
% Role subtitle
  \node[anchor=south west,\backgroundfontcolor, align=left] at ({0.4}, \cardheight-0.9) {{{\normalfont\mdseries \small You are a}}};%
% Role title
  \node[anchor=south west,\backgroundfontcolor, align=left] at ({0.4}, \cardheight-1.5) {{\uppercase{\normalfont\bfseries\itshape \LARGE #1}}};%
% Seperator line
  \draw [very thin, materialgrey] (0.4,\cardheight-1.6) -- (\cardwidth-0.4,\cardheight-1.6);
% Role description title
  \node[anchor=north west, align=left,\backgroundfontcolor,font=\normalsize, text width=(\cardwidth - 1)*1cm] at ({0.4}, \cardheight-2){
      \normalfont\bfseries #2\\
    };%
% Role description content
  \node[anchor=north west, align=left,\backgroundfontcolor,font=\normalsize, text width=(\cardwidth - 0.9)*1cm] at ({0.4}, \cardheight-2.6){
    \setlength{\parskip}{1em}\normalfont\mdseries #3%
  };%
% Seperator line
  \draw [very thin, materialgrey] (0.4,1.6) -- (\cardwidth-0.4,1.6);
% Team Info
  \node[anchor=north west, align=left,\backgroundfontcolor,font=\small] at ({0.4}, 1.4){
    {\normalfont\bfseries Allegiance}\\{\normalfont\mdseries #4}};
% Wake Info
  \node[anchor=north east, align=left,\backgroundfontcolor,font=\small] at ({\cardwidth-0.4}, 1.4){
    {\normalfont\bfseries Wake at night?}\\{\normalfont\mdseries #5}
      };%
\end{scope}
\end{tikzpicture}
}


\newcommand{\condition}[1]{
``#1''
}
\newcommand{\character}[1]{
\textbf{#1}
}
